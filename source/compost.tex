\documentclass[../main.tex]{subfiles}
\begin{document}

\chapter{El compostaje}

El proceso biológico de transformación de la materia orgánica de los residuos se llama compost, más conocido como abono para el suelo. \\
El compost o abono es un material orgánico que se puede agregar al suelo para ayudar a las plantas a crecer. Los restos de comida y los desechos del jardín juntos representan actualmente más del 30 por ciento de lo que tiramos, en su lugar podrían convertirse en abono. La fabricación de abono mantiene estos materiales fuera de los basureros donde ocupan espacio y liberan metano, un potente gas de efecto invernadero.\\

\begin{recuadroV}
    Los beneficios que nos proporciona transformar los desperdicios orgánicos al convertirlos en compost son muchos; enriquece el suelo, ayuda a retener la humedad, suprime las enfermedades y plagas de las plantas, reduce la necesidad de fertilizantes químicos, fomenta la producción de bacterias y hongos beneficiosos que descomponen la materia orgánica para crear humus, un material rico en nutrientes, reduce nuestra huella de carbono.
\end{recuadroV}

El compostaje reduce significativamente la cantidad de basura en un relleno sanitario y baja los costos y las emisiones de carbono que se necesitan para transportar y procesar esos materiales. 
También el compostaje enriquece el suelo con nutrientes, lo que disminuye la necesidad de fertilizantes y pesticidas. Los fertilizantes y pesticidas requieren combustibles fósiles para su producción y envío, y algunos de ellos son potencialmente dañinos para nuestra salud.\\

El abono aumenta la capacidad del suelo para retener la humedad, lo que ayuda a prevenir la erosión al reducir el escurrimiento de la corriente de lluvia. Y el compost previene y suprime las enfermedades y plagas de las plantas.
El compostaje puede ayudar a secuestrar carbono, lo que significa que puede ayudar a eliminar el carbono de la atmósfera.\\

Las plantas crecen más rápidamente en suelo suplementado con compost, lo que significa que pueden extraer más dióxido de carbono del aire. La cantidad de carbono secuestrado en el suelo y las plantas después de la aplicación de compost húmedo podría reducir significativamente los gases de efecto invernadero si se aplica a gran escala.


\section{Concepto}

Todo el compostaje requiere los siguientes ingredientes básicos:

\begin{description}
    \item [Alimento] 
    Materiales marrones como hojas muertas, ramas y ramitas, y verdes como recortes de césped, desechos vegetales, restos de frutas y café.\\[-5pt]
    \item [Oxígeno y Nitrógeno] 
    La cantidad de oxígeno dentro de la pila de compost también es importante, ya que un déficit de oxígeno hace que los microorganismos anaeróbicos se apoderen de ellos y eso puede llevar a una pila de compost apestosa. Se puede agregar oxígeno a la pila de compost revolviendo o volteando la pila, con un temperatura exterior ambiente.\\[-5pt]
\end{description}

Tener la cantidad adecuada de agua, alimentos verdes y marrones, aire y temperatura es importante para el desarrollo del compost.

\subsection{Tipos de compostaje}

Antes de comenzar a acumular, tenemos que saber que hay dos tipos de compostaje: en frío y en caliente. \\

El \tcbox{compostaje frío} es tan simple como recolectar los desechos del jardín o sacar los materiales orgánicos de la basura, como cáscaras de frutas y vegetales, posos de café y cáscaras de huevo, y luego ponerlos en una pila o contenedor. En el transcurso de un año aproximadamente, el material se descompondrá.\\

El \tcbox{compostaje en caliente} requiere más trabajo, pero es un proceso más rápido; obtendremos abono en uno a tres meses durante el clima cálido. Se requieren cuatro ingredientes para el compost caliente de cocción rápida: nitrógeno, carbono, aire y agua. Juntos, estos elementos alimentan a los microorganismos que aceleran el proceso de descomposición. En primavera u otoño, cuando los desechos del jardín son abundantes, se puede mezclar una gran cantidad de compost y luego comenzar una segunda mientras la primera se \enquote{cocina}.\\

El \tcbox{lombricompost} se elabora con la ayuda de lombrices. Cuando estos gusanos se comen los restos de comida, liberan moldes, que son ricos en nitrógeno. Sin embargo, no puede se usar cualquier gusano viejo para esto: necesitamos lombrices para compostaje. Las lombrices se  pueden comprar.

\subsection{Materiales compostables}

\subsubsection{Materiales \emph{verdes}}

\begin{itemize}
    \item Frutas y vegetales.
    \item Cáscaras de huevo preferentemente trituradas.
    \item Café usado, yerba, té sin su bolsita preferiblemente o al menos sin la etiqueta.
    \item Cáscaras de frutas y nueces.
    \item Periódico triturado.
    \item Recortes de césped.
    \item Plantas de interior.
\end{itemize}

Pensar dos veces antes de agregar cáscaras de cítricos, cebollas y ajo a la pila de abono casero. Se cree que estos materiales repelen las lombrices de tierra, que son una parte vital del jardín.

\subsubsection{Materiales \emph{marrones}}

\begin{itemize}
    \item Hojas, flores y plantas marchitas o secas. 
    \item Hierba seca.
    \item Semillas y carozos de frutas frescas.
    \item Restos de poda triturados.
    \item Cenizas y aserrín de madera natural.
    \item Restos de cosecha de la huerta.
    \item Hueveras de cartón o cartón ondulado.
    \item Fruta caída.
    \item Tubo de cartón del papel de aluminio, papel de cocina, papel higiénico, etc.
    \item Hojas.
    \item Aserrín.
    \item Astillas de madera.
\end{itemize}

\subsection{Materiales no compostables}

\begin{description}
    \item [Alimentos de origen animal] 
    Lácteos, carnes, huesos, conservas, escabeches; algunos tienen elevados contenidos de sales.  Los productos de origen animal, como la carne y los lácteos, se pueden convertir en abono, pero a menudo generan malos olores y atraen plagas como roedores e insectos.\\[-5pt]
    \item [Recortes de jardín tratados con pesticidas químicos] 
    Pueden matar organismos beneficiosos para el compostaje.\\[-5pt]
    \item [Revistas o papel impreso en color] 
    Los productos de papel natural son compostables, pero se deben evitar los papeles brillantes, ya que pueden abrumar la tierra con productos químicos que tardan más en descomponerse.\\[-5pt]
    \item [Cenizas de carbón] 
    Las cenizas \emph{de carbón} contienen azufre y hierro en cantidades lo suficientemente altas como para dañar las plantas.\\[-5pt]
    \item [Restos sanitarios como pañuelos y papel higiénico] 
    Pueden contener microorganismos patógenos vectores de enfermedades.\\[-5pt]
    \item [Desechos con presencia de químicos] 
    Filtros de cigarrillo, pañales, toallitas, tampones, medicinas, etc.\\[-5pt]
    \item [Biomasa] 
    Excremento de humanos, desechos de animales, especialmente heces de perros y gatos, generan olores no deseados y pueden contener parásitos. El estiércol y boñiga está permitido ya que está completamente basado en material vegetal compostable.\\[-5pt]
    \item [Materiales inorgánicos] 
    Plástico, vidrio, aluminio, latas, maderas compuestas (MDF, aglomerado, prensada, etc.). ¡Mejor los reciclemos si podemos!\\[-5pt]
\end{description}

\section{El proceso de compostaje}

\subsection{Preparativos y armado del compostaje}

Para hacer nuestra propia pila de abono caliente, si podemos, empezaremos en tierra desnuda. Esto permitirá que las lombrices y otros organismos benéficos aireen el abono. Con el lugar designado podemos abordar la construcción del compostaje.\\

\begin{description}
    \item [Estructura inferior] 
    Colocaremos \tcbox{ramitas o paja primero}, a unos centímetros de profundidad. Esto ayudará al drenaje y ayudará a airear la pila.\\[-5pt]
    \item [Adición de material] 
    Agregaremos los materiales de \tcbox{abono en capas}, alternando húmedo y seco. Recordemos que los ingredientes húmedos son restos de comida, café, yerba, entre otros. Los materiales secos son paja, hojas, aserrín y cenizas de madera.\\[-5pt]
    \item [Adición de nitrógeno] 
    Agreguemos estiércol (menos de perro o gato), abono verde (trébol, pasto, recortes de jardinería frescos) o \tcbox{cualquier fuente de nitrógeno}. Esto activa la pila de abono y acelera el proceso.\\[-5pt]
    \item [Balanceo de materiales] 
    Mezclaremos \tcbox{tres partes de material marrón} con \tcbox{una parte de materiales verdes}. Si la pila se ve demasiado húmeda y huele, le pondremos más elementos marrones o tendremos que airear con más frecuencia. Si se ve extremadamente marrón y seco, agregaremos elementos verdes y agua para humedecerlo un poco.\\[-5pt]
    \item [Asegurando humedad y calor] 
    \tcbox{Cubrirlo} con todo lo que se tenga: maderas, láminas de plástico, restos de alfombra. Cubrir ayuda a retener la humedad y el calor, dos elementos esenciales para el abono sobre todo la humedad, que es muy importante tener bajo control. Cubrir también evita que la lluvia riegue demasiado el compost. El abono debe estar húmedo, pero no empapado de lo contrario los microorganismos en su pila se encharcarán y se ahogarán. Si regamos debe ser controlado dado que la  pila debe tener \tcbox{la consistencia de una esponja húmeda}. Hay que controlar la temperatura del compost con un termómetro o con la mano, en el medio de la pila, para asegurarse que los materiales se descompongan correctamente. Los materiales deben estar calientes.\\[-5pt]
\end{description}

\subsection{Mantenimiento}

\subsubsection{La regularidad}

Si el abono huele mal agregar más hojas, papel marrón o recortes de césped (los marrones) para equilibrar los restos de frutas y verduras que causan el mal olor a medida que se descomponen. Seguir agregando marrones y verdes, cuando el volumen se vuelva difícil de manejar, hay que comenzar una nueva pila y dejar que la pila vieja termine (descomponga para que quede completamente negra y no tenga grandes trozos de desperdicio de comida reconocible) antes de agregarla a suelo.\\

Durante la temporada de crecimiento \tcbox{se debe proporcionar oxígeno a la pila} girándola una vez a la semana con una herramienta de jardín. El mejor momento para darle la vuelta al abono es cuando el centro de la pila se siente caliente o cuando un termómetro marca entre 50 °C y 65 °C. Revolver la pila ayudará a que se cocine más rápido y evitará que el material se enrede y desarrolle olor. En este punto, las capas han cumplido su propósito de crear cantidades iguales de materiales verdes y marrones en toda la pila, así que hay que revolver bien. \tcbox{Mezclar o voltear la pila de compost} es clave para airear los materiales de compostaje y acelerar el proceso hasta su finalización.\\

\begin{recuadroV}
    Además de airear con regularidad, es conveniente picar y triturar los ingredientes crudos en tamaños más pequeños para acelerar el proceso de compostaje.
\end{recuadroV}

\subsubsection{El balance}

El secreto para una pila de abono saludable: la relación entre carbono ({\color{CompostGreen!50!black}C}) y nitrógeno ({\color{CompostGreen!50!black}N}).\\

La materia rica en carbono le da al compost su cuerpo ligero y esponjoso.

\begin{table}[H]
    \centering\sffamily
    {%
    \def\arraystretch{1.5}%
    \resizebox*{\textwidth}{!}{
    \begin{tabular}{cccccc}
        \rowcolor{CompostGreen!25}

         Ramas & Tallos & Hojas secas & Cáscaras & Trozos de madera & Polvo de cortezas\\
         Aserrín & \multicolumn{2}{c}{Bolsas de papel marrón trituradas}  & Tallos de maíz & Café usado & Coníferas\\
        \rowcolor{CompostGreen!25}

        \multicolumn{2}{c}{Cásaras de huevo desarmadas} & Paja o heno & Ceniza de madera & \multicolumn{2}{c}{Filtros de café no plásticos} \\
    \end{tabular}
    }
    \caption*{\sffamily\color{CompostGreen!50!black}Ejemplos de materia rica en carbono}
    }
    \label{carbono1}
\end{table}%

\begin{recuadroR}
    Una pila de abono saludable debería tener mucho más carbono que nitrógeno.
\end{recuadroR}

Por su parte, el nitrógeno o la materia rica en proteínas proporcionan materias primas para la fabricación de enzimas.

(estiércol, restos de comida, recortes de césped verde, desechos de cocina y hojas verdes) 

\begin{table}[H]
    \centering\sffamily
    {%
    \def\arraystretch{1.5}%
    \resizebox*{0.5\textwidth}{!}{
    \begin{tabular}{ccc}
        \rowcolor{CompostGreen!25}

          Desechos de cocina & Restos de comida & Hojas verdes \\
          Estiércol & \multicolumn{2}{c}{Recortes de césped o pasto verde}\\
    \end{tabular}
    }
    \caption*{\sffamily\color{CompostGreen!50!black}Ejemplos de materia rica en nitrógeno}
    }
    \label{carbono1}
\end{table}%

\begin{recuadroV}
    Una regla general simple es usar un tercio de materiales verdes y dos tercios de materiales marrones. El volumen de los materiales marrones permite que el oxígeno penetre y alimente a los organismos que residen allí.
\end{recuadroV}

Demasiado nitrógeno produce una masa anaeróbica densa, maloliente y que se descompone lentamente. Una buena higiene del compostaje significa cubrir el material fresco rico en nitrógeno, que puede liberar olores si se expone al aire libre, con material rico en carbono, que a menudo emana un olor fresco y maravilloso. \\

\begin{recuadroR}
    \centering
    {¡En caso de duda agreguemos carbono!}
\end{recuadroR}

\subsection{Usando el compost}
\centering
\textbf{¡Alimentemos el jardín!}\\[8pt]
\raggedright

Cuando el compost ya no emite calor y se vuelve seco, marrón y quebradizo, está completamente cocido y listo para alimentar a nuestras plantitas. \\



Algunos jardineros preparan lo que se conoce como té de compost con compost terminado. Esto implica permitir que el abono completamente formado se \enquote{empape} en agua durante varios días y luego colarlo para usarlo como fertilizante líquido casero.\\
\begin{table}[H]
    \centering\sffamily
    {%
    \def\arraystretch{1.5}%
    \resizebox*{\textwidth}{!}{
    \newcolumntype{L}{>{\small\raggedright\arraybackslash}X}
    \begin{tabularx}{1.3\textwidth}{LLLLL}
        Nuestro entorno & Nuestra disponibilidad & \multicolumn{3}{c}{¿Qué compostaremos con mayor frecuencia?}\\
        \toprule
        & & Residuos de cocina & Residuos de cocina y de jardín & Abundantes residuos de jardín\\
        \rowcolor{CompostGreen!25}

         Urbano & Sin espacios abiertos & Lombricompuesto & \centering\arraybackslash--- & \centering\arraybackslash---\\
         Urbano & Con mínimo espacio abierto, como patio o balcón & Lombricompuesto o tambor de compostaje & Tambor de compostaje & \centering\arraybackslash---\\
        \rowcolor{CompostGreen!25}

        Suburbano & Con jardín & Contenedor cerrado o tambor de compostaje & Contenedor cerrado o tambor de compostaje & Contenedor cerrado o casero\\
        Rural & Con amplio espacio abierto & Contenedor cerrado o tambor de compostaje & Pila de compost abierta, contenedor cerrado o tambor de compostaje & Pila de compost abierto o múltiples contenedores cerrados\\
    \end{tabularx}
    }
    \caption*{\sffamily\color{CompostGreen!50!black}Tipos de compost más usados según dónde nos encontremos}}
    \label{distanciasiembra1}
\end{table}%

\subsubsection{Té con moho de hojas}

\begin{recuadroV}
    Usemos hojas para hacer un \enquote{té} nutritivo para sus plantas. Simplemente envolviendo un pequeño montón de hojas en arpillera y sumergiéndolo en un balde grande de agua. Dejar actuar durante tres días, luego retirar la \enquote{bolsita de té} y verter el contenido en el abono. Sacar el agua enriquecida con un balde más pequeño y usemos este líquido para regar plantas y arbustos.
\end{recuadroV}

El compost es increíblemente fácil de hacer y respetuoso con el medio ambiente. Además, es un placer para nuestro jardín. Con solo algunas sobras de cocina y algo de paciencia, tendremos el jardín más feliz posible.

\pagebreak


\end{document}