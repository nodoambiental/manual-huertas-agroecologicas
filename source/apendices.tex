\documentclass[../main.tex]{subfiles}
\begin{document}

% ----- Abre página apaisada para la tabla -----

\newgeometry{margin=2cm,bottom=3cm} \pagebreak

\begin{landscape}%
\section{Datos técnicos sobre la cosecha}\label{appendixA}%
    \begin{table}[H]
        \centering\sffamily
        {%
        \def\arraystretch{1.5}%
        \resizebox*{!}{0.9\textheight}{
        \begin{tabular}{lccl}
            Cultivos & Distancia entre hileras {\color{black!50}[cm]} & Distancia entre plantas {\color{black!50}[cm]} & Ciclo vegetativo {\color{black!50}[Días]}\\
            \toprule
            \rowcolor{CompostGreen!25}

            Acelga & 20 & 20 & 65\\
            Ají & 35 & 35 & 180 a 190\\
            \rowcolor{CompostGreen!25}

            Ajo & 10 & 10 & 150\\
            Arveja &  30 & 30 & 70 a 90\\
            \rowcolor{CompostGreen!25}

            Cebolla & 10 & 10 & 120 a 150\\
            Haba & 20 & 20 & 180 a 200\\
            \rowcolor{CompostGreen!25}

            Lechuga & 20 & 20 & 60\\
            Papa & 30 & 30 & 90 a 120\\
            \rowcolor{CompostGreen!25}

            Pepino & 30 & 30 & 120 a 150\\
            Pimiento & 30 & 30 & 80 a 100\\
            \rowcolor{CompostGreen!25}

            Remolacha & 20 & 20 & 75\\
            Repollo & 35 & 35 & 60 a 90\\
            \rowcolor{CompostGreen!25}

            Tomate & 25 & 25 & 80 a 90\\
            Zanahoria & \multicolumn{2}{c}{Chorro continuo} & 80 a 120\\
            \rowcolor{CompostGreen!25}

            Coliflor & 20 & 20 & 120 a 150\\
            Espinaca & 10 & 30 & 80 a 90\\
            \rowcolor{CompostGreen!25}

            Maíz & 10 & 30 & 90 a 120\\
        \end{tabular}
        \begin{tabular}{lc}
            Especie & Años\\
            \toprule
            \rowcolor{CompostGreen!25}

            Acelga & 4\\
            Apio & 5\\
            \rowcolor{CompostGreen!25}

            Calabaza & 5\\
            Cebolla &  1\\
            \rowcolor{CompostGreen!25}

            Col & 4\\
            Espinaca & 4\\
            \rowcolor{CompostGreen!25}

            Arveja & 3\\
            Haba & 4\\
            \rowcolor{CompostGreen!25}

            Lechuga & 3\\
            Nabo & 4\\
            \rowcolor{CompostGreen!25}

            Pimiento & 3\\
            Pepino & 5\\
            \rowcolor{CompostGreen!25}

            Rábano & 4\\
            Zanahoria & 3\\
            \rowcolor{CompostGreen!25}

            Tomate & 3\\
        \end{tabular}
        }
        \caption*{
            {\raggedright\sffamily\color{CompostGreen!50!black}Distancia de siembra de algunas hortalizas}\hfill{\raggedleft\sffamily\color{CompostGreen!50!black}Viabilidad de las semillas}
        }
        }
        \centering\observation{¿Por qué las tablas son incongruentes? Hay especies que no están en un lado y en otro sí }
        \label{distanciasiembra1}
    \end{table}%
\thispagestyle{empty}
\clearpage
\end{landscape}

\restoregeometry
\flushleft

% ----- Retoma el contenido regular ----- 

\section{Sistema de riego por goteo}\label{appendixB}

Utilizando pequeñas cantidades de agua, esta tecnología contribuye a disminuir el estrés hídrico causado por la falta de lluvia y la excesiva evapotranspiración producida por las altas temperaturas. 
Esto es muy importante ya que en la zona el agua apta para riego y consumo humano no es abundante. El sistema permite mantener un nivel de humedad constante, sin que se produzcan fluctuaciones bruscas en el contenido de agua en el suelo. \\

\subsubsection{¿Qué es?}

Consiste en botellas plásticas descartables, lavadas, a las que se les hace una perforación pequeña en la base, de aproximadamente 2mm. \\
Son llenadas con agua y se las tapa. Al taparlas, el vacío formado entre el agua y la tapa ejerce una fuerza contraria al propio peso del agua, de manera tal que controlando la hermeticidad del vacío en la superficie de la botella podamos controlar el caudal de salida. Si, con cuidado, vamos abriendo la tapa, permitimos que el caudal se incremente, y viceversa.\\
\observation{La explicación fué cambiada porque era incorrecta, checar}
Las botellas se pueden colgar a una altura que resulte cómoda para su manipulación, con un hilo o un alambre alrededor del pico que facilite su reemplazo. También pueden ser enterradas, a una profundidad de no más de 15 a 20 cm, y deberían ubicarse en la entrelinea de la huerta, de esta manera se riega la parte más necesaria que son las raíces.  \\

El riego tradicional utiliza una cantidad mucho mayor de agua, por lo que empleando este sistema podemos mantener el suelo húmedo durante todo el ciclo del cultivo, evitando los problemas generados por los cambios abruptos de humedad del suelo. Esto puede presentar una ventaja importante, ya que la disponibilidad de agua en la zona puede ser una dificultad. Además, si se acompaña de otras medidas, como por ejemplo el agregado de materia orgánica al suelo, se puede evitar la salinización de los suelos que ocasiona el sistema convencional. \\

La recarga de las botellas debería hacerse aproximadamente cada día por medio, y en caso de lluvias, bloqueamos del todo el flujo de agua (cerrando las tapas por completo). \\

\subsubsection{Implementando el sistema}

Se requieren \\
\begin{itemize}
    \item Botellas plásticas descartables usadas
    \item Una aguja de coser grande o un clavo finito
    \item Hilo, cinta o alambre
    \item Marcador permanente que resista el sol
\end{itemize}

\hfill\\

Un método de fabricación\\
\begin{enumerate}
    \item Lavar bien las botellas y asegurarse que no tengan roturas o pérdidas.
    \item Calentar la aguja o clavo usando una llama (como una hornalla o un encendedor).
    \item Usar el metal caliente para perforar el fondo de la botella, apoyándolo en el plástico y empujando (el calor derrite el plástico y la presión que realizamos permite realizar la perforación). Hay que tener cuidado de respetar el tamaño deseado de los agujeros (2 mm).
    \item Una vez fría, llenamos de agua la botella y con su tapita la cerramos del todo.
    \item Abrimos lentamente la tapita, hasta verificar que caiga más o menos una gota por segundo (o lo que pueda requerir el cultivo).
    \item En ese punto, marcar la tapita y el cuerpo de la botella para tener una referencia y saber en qué posición debe estar, ya que cada vez que la recarguemos hay que abrirla de nuevo.
    \item Con el hilo o alambre podemos construir un gancho o soporte alrededor del pico, para colgarla o manipularla.
\end{enumerate}

\section{Preparados vegetales para proteger las plantas}\label{appendixC}

\subsection{Purín de ortiga}

El purín de ortiga es el producto de la fermentación de la planta de ortiga (Urtica sp.) la cual tiene utilidad en el mantenimiento de la salud de los cultivos. \\

\subsubsection{¿Qué es?}

\observation{No explica físicamente qué es un purín}

Según el grado de maduración, es un producto repelente, insecticida, fungicida, fitoestimulante o activador del suelo, el compost y es un “llamador” de lombrices. \\

Es rico en calcio, nitrógeno y potasio. Actúa como un abono de crecimiento, es un estimulador de la vitalidad y enriquece la vida bacteriana del suelo. Las plantas a las que se aplica purín de ortigas se vuelven más fuertes contra el ataque de hongos y plagas.  \\

\subsubsection{¿Cómo se hace?}

Se requieren \\
\begin{itemize}
    \item Contenedor plástico de agua de 10 litros
    \item Mucha ortiga
    \item Agua
\end{itemize}

\hfill\\

Preparación\\
\begin{enumerate}
    \item Recolectar las partes aéreas de la planta de Ortiga.
    \item Triturar las hojas y tallos para facilitar la descomposición. 
    \item Llenar el contenedor con ortiga hasta la mitad, y luego incorporar agua hasta llenarlo.
    \item Dejar reposar a la sombra y con la boca tapada con media sombra así no se llena de basura. 
    \item Revolver el preparado cada cierto tiempo.
    \item A los 7 dias agregar los 5 kg de ceniza, que ayudara a eliminar el olor de la mezcla y a mejorar la composición del fertilizante por la incorporación de potasio presente en la ceniza.
    \item Filtrar el preparado.
\end{enumerate}

\subsubsection{¿Cuándo y cómo lo usamos?}

\begin{table}[H]
    \centering\sffamily
    {%
    \def\arraystretch{1.5}%
    \resizebox*{\textwidth}{!}{
    \begin{tabular}{lcr}
        Tiempo de fermentación {\color{black!50}[Días]} & Uso & Modo de uso\\
        \toprule
        \rowcolor{CompostGreen!25}

        Aproximadamente 3 & Repelente e insecticida & Pulverización en hojas\\
        Entre 15 y 25 & Fertilizante & Pulverización en hojas\\
        \rowcolor{CompostGreen!25}

        30 o más & Enriquecedor del suelo y del compost & Por goteo\\
    \end{tabular}
    }
    \caption*{\sffamily\color{CompostGreen!50!black}Usos según el tiempo de fermentación}
    }
    \label{purinortiga1}
\end{table}%
\hfill\\
Las cantidades de biofertilizantes dependen del tipo del cultivo, de las necesidades de nutrición de las plantas, de la etapa de desarrollo y del ciclo de las plantas (pre-floración, floración, fructificación, pos cosecha, desarrollo vegetativo y semillas, etc.).\\
Las concentraciones varían de 0,5 a 3 litros del purín en 20 litros de agua y dado que las plantas llevan adelante sus actividades (crecen, gastan energía, se reproducen y mas..) todos los días es recomendable realizar las aplicaciones con intervalos  cortos entre una aplicación y otra y en concentraciones muy bajas. \\

Las sustancias naturales de la ortiga se degradan con la luz, temperatura y aire (oxígeno) por lo que debe aplicarse temprano por la mañana o a la tarde, cerca de la caída del sol. 
La aplicación del purín de ortiga en los cultivos es foliar o por riego. 

\subsection{Purín de cebolla}

La cebolla contiene propiedades que son muy beneficiosas a la hora de combatir ciertas plagas y enfermedades, y tiene importantes acciones antifúngicas, insecticidas y bactericidas. 

\subsubsection{¿Qué es?}

El preparado de cebolla nos ayudará a prevenir y combatir la plaga de pulgón, mosca de la zanahoria y enfermedades de hongos como el mildiu y la roya. 
No sólo podemos realizar preparados con la cebolla para aplicarlos sobre los cultivos. Además podemos cultivarlas en el huerto y jardín para crear sinergias (una acción conjunta) con otras plantas y repeler la aparición de plagas en las plantas colindantes. 

\subsubsection{¿Cómo se hace?}

Se requieren \\
\begin{itemize}
    \item Contenedor plástico de 10 litros
    \item Más de un kilo de cebolla
    \item Agua
\end{itemize}

\hfill\\

Preparación\\
\begin{enumerate}
    \item Cortamos en cuatro o seis trozos las cebollas (dependiendo de lo grandes que sean).
    \item Machacamos los trozos con la ayuda de un mortero. 
    \item Las echamos en el contenedor, y luego se incorpora agua hasta llenarlo.
    \item Dejar reposar por 2 semanas manteniendo tapado y en un lugar a la sombra.
    \item Revolver el preparado cada cierto tiempo. 
    \item Filtrar el preparado.
\end{enumerate}

\subsubsection{¿Cuándo y cómo lo usamos?}

Se recomiendan diferentes dosis de aplicación dependiendo del cultivo y la enfermedad o plaga a curar. 

Puede usarse para prevenir el moho gris en frutillas, aplicándolo pulverizado en disoluciones de 2 litros de purín en 20 litros de agua. 

Para prevenir ataques de la mosca de la zanahoria puede aplicarse diluido en el agua de riego en una proporción de 1 litro de purín en 20 litros de agua. 

Para prevenir podredumbres puede aplicarse en el agua de riego en una proporción de 2 litros de purín en 20 litros de agua. 

Para prevenir ataques de hongos en frutales pulverizar en la copa con disoluciones de 1 litro de purín en 20 litros de agua. 

En árboles frutales, pepinos y tomates las partes atacadas se pulverizan con el purín sin diluír.

\section{\color{black!50}Composteras}

\observation{Este apéndice todavía no existe}



\end{document}